\newpage
\section{Appendix}
%\thispagestyle{empty}
%\pagenumbering{Roman} 	% R XI and r xi
%\addcontentsline{toc}{section}{Appendix}

%%% table settings
\setcounter{table}{0}
\renewcommand{\thetable}{A\arabic{table}}

\setcounter{figure}{0}
\renewcommand{\thefigure}{A\arabic{figure}}


% \setcounter{table}{1}
% \renewcommand{\thetable}{B\arabic{table}}
% \setcounter{figure}{1}
% \renewcommand{\thefigure}{B\arabic{figure}}

\textbf{Ethical Position Questionnaire - German}

Questions were asked in random order.

Bitte geben Sie das Ausmaß Ihrer persönlichen Zustimmung oder Ablehnung an.

eqp1: Das Wohl anderer zu opfern, ist niemals wirklich notwendig.

eqp2: Moralische Standards sollten als etwas Individuelles gesehen werden: was eine Person als moralisch ansieht, kann eine andere als unmoralisch bewerten.

eqp3: Die Würde und das Wohlergehen der Menschen sollten die wichtigste Sorge in jeder Gesellschaft sein.

eqp4: Ob eine Lüge als unmoralisch oder sogar moralisch zu beurteilen ist, hängt ganz von den Umständen ab.

eqp5: In sozialen Beziehungen sind ethische Probleme oft so komplex, dass man Personen erlauben sollte, ihre eigenen persönlichen Regeln zu finden.

eqp6: Was "ethisch" ist, variiert zwischen Situationen und Kulturen.

eqp7: Es ist unmoralisch, negative Folgen einer Handlung durch positive Folgen verrechnen zu wollen.

eqp8: Man darf andere Personen weder psychisch noch physisch schädigen.

eqp9: Wenn eine Handlung eine unschuldige Person schädigen könnte, muss man sie unterlassen.

eqp10: Es gibt keine ethischen Prinzipien, die so wichtig sind, dass sie eine allgemeingültige Vorschrift bilden könnten.

eqp11: Moralisches Handeln liegt dann vor, wenn es der Ideal- Handlung entspricht.

eqp12: Man darf keine Handlungen ausführen, die in irgendeiner Weise die Würde und das Wohlergehen anderer Personen bedrohen.

eqp13: Risiken in Kauf zu nehmen, die andere Personen betreffen, ist nicht tolerierbar, egal wie gering sie sind.

eqp14: Potentielle Schädigungen Dritter in Kauf zu nehmen, ist immer schlecht, egal welche guten Zwecke verfolgt werden.

eqp15: Moralische Standards sind jeweils persönliche Regeln, sie sollten nicht auf die Beurteilung anderer angewendet werden.

eqp16: Die Frage, was ethisch richtig ist, wird sich niemals beantworten lassen, da es sich bei der Entscheidung, was moralisch oder unmoralisch ist, um eine persönliche Entscheidung handelt.

eqp17 Eine starre Ethik-Vorschrift, die bestimmte Handlungsmöglichkeiten verhindern soll, kann der Verbesserung sozialer Beziehungen sogar im Wege stehen.

eqp18: Man sollte sichergehen, mit seinen Handlungen niemanden zu verletzen oder zu schädigen.

eqp19: Verschiedene Arten von Moral dürfen nicht als mehr oder weniger "Gut" bewertet werden.

eqp20: Über das Lügen lässt sich keine Regel formulieren; ob eine Lüge zulässig ist oder nicht, hängt von der Situation ab.


\newpage

\textbf{Treatment - Information Group and starting Arguments for discussion group}

\textit{Argument gegen Weiche umstellen und unbeteiligte Person vor den Waggon zu stossen}

Jedem Menschen ist ungeachtet seiner Herkunft, seines sozialen Standes, etc. eine Würde zu eigen, die unantastbar ist. Wenn man Menschenleben gegeneinander aufwiegt, wird einem Menschenleben zwangsläufig ein bestimmter Wert zugemessen. Dieser Wert kann jedoch niemals für einen Menschen bestimmt werden. Wer entscheidet, wie dieser Wert bemessen werden soll und wo wird dabei die Grenze gezogen? Wenn man denkt es sei richtig ein Menschenleben zu opfern, um dafür mehrere Menschen zu retten, könnte man in einem zugespitzten Beispiel einem gesunden Menschen lebenswichtige Organe entnehmen um einige Patienten, die dringend einer Organspende bedürfen, zu retten.
Der Mensch ist ein vernunftbegabtes Wesen, was ihn zur Freiheit befähigt. Es entspricht daher der sittlichen Pflicht, einen jeden Menschen zu achten und als Zweck an sich zu behandeln. Wenn man einen Menschen als bloßes Mittel zum Nutzen anderer instrumentalisiert, wird er zum Objekt degradiert und in seiner Würde verletzt. Individuelle Rechte müssen geachtet werden und dürfen auch nicht zur Maximierung kollektiven Wohls beschnitten werden.
Bei einer Geiselnahme in Frankreich hat sicher Polizist Arnaud Betrame für eine weibliche Geisel austauschen lassen. Durch den Geiselnehmer wurde Betrame lebensgefährlich verletzt und erlag später seinen Verletzungen. Die weibliche Geisel hat heute große psychologische Probleme und lebt mit einer schweren Schuld. Eine Rettung vieler Menschen kann auch für die Geretteten schwere psychologische Folgen haben.

\textit{Argumente für Weiche umstellen und unbeteiligte Person vor den Waggon zu stossen}

Menschenleben werden tagtäglich gegeneinander aufgewogen. Wenn beispielsweise eine Mutter durch die Geburt ihres Kindes sterben würde, ist es erlaubt stattdessen das Kind zu töten. Gerade weil es relevant ist Menschenleben zu achten, ist es richtig, dies bei der größtmöglichen Anzahl zu tun. Es gibt viele Beispiele, die das vor Augen führen, z. B. das von Graf von Stauffenberg, der den Tod weniger Menschen in Kauf nahm, um tausende Menschenleben zu retten.
Beim Abwägen von Handlungsoptionen sollte man versuchen den größtmöglichen Nutzen für die größtmögliche Anzahl an Menschen, beziehungsweise die Gesellschaft umzusetzen und das kollektive Wohl zu steigern/schützen. Daher sollte das Gemeinwohl die ausschlaggebende Größe bei der Beurteilung von Handlungen sein, selbst wenn dadurch die Rechte bestimmter Individuen als kleineres Übel beschnitten werden müssen.
Nichts tun ist auch eine Entscheidung, die moralische Konsequenzen hat. Man kann folglich nicht sagen, dass es besser ist, nichts zu tun.


\textbf{Question Wording Dependent Variable - Scenario 1}

Bitte stellen Sie sich nun folgende Situation vor:

Ein losgelöster Waggon bewegt sich auf fünf Personen zu, die sich auf den Gleisen befinden und getötet werden, wenn nichts unternommen wird. Anna kann eine unbeteiligte Person vor den Waggon stoßen. Der Waggon würde durch den Zusammenstoß mit der unbeteiligten Person gestoppt werden. Dadurch würden die fünf Personen gerettet, aber die unbeteiligte Person getötet.
Wie sehr ist es moralisch vertretbar, dass Anna den Waggon auf diese Weise stoppt?

Antwortkategorien: Skala von 1 “Gar nicht” über  6 “Teils/Teils” bis 11 “Absolut”


\textbf{Question Wording Dependent Variable - Scenario 2}

Bitte stellen Sie sich nun folgende Situation vor:

Ein losgelöster Waggon bewegt sich auf fünf Personen zu, die sich auf den Gleisen befinden und getötet werden, wenn nichts unternommen wird. Anna kann eine unbeteiligte Person vor den Waggon stoßen. Der Waggon würde durch den Zusammenstoß mit der unbeteiligten Person gestoppt werden. Dadurch würden die fünf Personen gerettet, aber die unbeteiligte Person getötet.
Wie sehr ist es moralisch vertretbar, dass Anna den Waggon auf diese Weise stoppt?

Antwortkategorien: Skala von 1 “Gar nnicht” über 6 “Teils/Teils” bis 11 “Absolut”



%---------------------------------------------------------------------------%
% Eigenständigkeiterklärung
%---------------------------------------------------------------------------%
%\clearpage
%\section*{Eigenständigkeitserklärung}
%\vspace*{2cm}
%\begin{center}
%	\begin{minipage}[t]{0.8\textwidth}
%		Hiermit versichern wir, dass wir die vorliegende Hausarbeit selbständig und nur mit den angegebenen Hilfsmitteln verfasst haben. Alle %Passagen, die wir wörtlich als auch sinngemäß aus der Literatur oder aus anderen Quellen wie z. B. Internetseiten entnommen haben, sind %deutlich als Zitat mit Angabe der Quelle kenntlich gemacht.
%		
%		\vspace*{60mm}
%		Stuttgart, 30.09.2018
%	\end{minipage}
%\end{center}
%
%
%\end{document}

