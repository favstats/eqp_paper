
%<!-- Titleseite -->
\thispagestyle{empty}

%%% redefine \maketitle
\renewcommand{\maketitle}{
	\begin{titlepage}
		\begin{center}
			\setlength{\parskip}{0pt}
			
			%	    \begin{flushright}
			%	    \colorbox{darkgray}{\color{white}{\Large \textsf{\@headerimg}}}
			%             \end{flushright}
			\begin{multicols}{2}
				\flushleft
                {Prof. Dr. André Bächtiger\par}
				%{Seminar: Transformation of representative democracy\par}
				{Institute for Social Sciences\par}
				{Department of Political Theory and\\ Empirical Research of Democracy\par}
				\begin{flushright}
					\includegraphics[width=7cm]{images/logo_stuttgart.jpg}
				\end{flushright}
			\end{multicols}
			\vspace*{2mm}
			\center
			{\LARGE {Seminar Paper} \par}
			
			\vspace*{10mm}
			
			
			{\fontsize{22}{30} {\bfseries Ethical Positions and Decision Making} \par}
			\vspace*{1mm}
			{ \Large Examining the (gendered) effects of Ethical Positions on Moral Decision Making using the Trolley Experiment}
			\vspace*{10mm}
			

	\centering
	\begin{tabular}{@{}ccc@{}}

    Fabio Votta             & Marlon Schumacher    & Quynh Nga Nguyen   \\
		Student ID: 2891518     & Student ID: 2954594  & Student ID: 2949965   \\  \\
	
		Abigail Alexander-Haw   & Matthias Rosenthal   & Nora Rohr        \\
		Student ID: 3197952     & Student ID: 2882873  & Student ID: 2955027            
	\end{tabular}
	
	\begin{tabular}{@{}cc@{}}
	\\
		Annemarie Hertner      & Rosa Seitz  \\
		Student ID: 2947556    & Student ID: 2876533    \\
	\end{tabular}
			
			
			\vspace*{5mm}
			
			
			
			
			
			\vspace*{5mm}
			{Date of Submission: 21.11.2018 \par} %\date{xxx}
			
		\end{center}
		\vspace*{2mm}
		\begin{abstract}
			\justifying
			\noindent In this paper, the ethical decision-making is examined through a survey of action decisions with
three different treatment groups regarding the known trolley dilemma and footbridge dilemma. The
original dilemma was developed by Foot (1978) p. 2 and consists of a mental experiment in which
the driver a runaway tram is heading towards a group of five people and will kill these people, unless
the tram is diverted to a second track where only one person is standing. The driver has the choice
of doing nothing, in which case, five people will perish, or divert the tram, ensuring the death of the
person on the other track. This thought scenario is traditionally employed to evaluate deontological
and consequentialist reasoning mechanisms (cf. Kahane 2015: 557).

In this paper we will also consider the gender differences in ethical decision-making, as studies
involving trolley dilemmas have found that men are more inclined to make utilitarian decisions
than women (cf. Zamzow \& Nichols 2009: 371). Research of the past two decades has focussed on
these kind of gender differences with regard to moral decisions. However, literature reviewing the
interplay between gender and individual ethical positions – and how it influences moral choices – has
been “surprisingly scarce” (Donoho et al. 2012: 57). This paper will therefore contribute to existing
literature by considering the role of ethical positions when explaining gender related patterns in
moral decision making.
		\end{abstract}
	    \vspace*{2mm}
        \center		
        {\large {Seminar: Political Framing} \par}
		
		
		
	\end{titlepage}
}

%%% automated table of contents
\newcommand{\contents}{
	\newpage
	\thispagestyle{empty}
	\vspace{20mm}
	\tableofcontents
}



%%% Title page
\maketitle
\newpage
\contents
\clearpage
\listoffigures
\clearpage
\listoftables
\clearpage

%\clearpage
%
%%<!-- Inhaltsverzeichnisse -->
%\thispagestyle{empty}
%\setstretch{1.15}
%\tableofcontents
%\listoffigures
%
%\clearpage
%\setstretch{1.44}
%<!-- \onehalfspacing -->