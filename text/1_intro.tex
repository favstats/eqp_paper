\documentclass[]{article}
\usepackage{lmodern}
\usepackage{amssymb,amsmath}
\usepackage{ifxetex,ifluatex}
\usepackage{fixltx2e} % provides \textsubscript
\ifnum 0\ifxetex 1\fi\ifluatex 1\fi=0 % if pdftex
  \usepackage[T1]{fontenc}
  \usepackage[utf8]{inputenc}
\else % if luatex or xelatex
  \ifxetex
    \usepackage{mathspec}
  \else
    \usepackage{fontspec}
  \fi
  \defaultfontfeatures{Ligatures=TeX,Scale=MatchLowercase}
\fi
% use upquote if available, for straight quotes in verbatim environments
\IfFileExists{upquote.sty}{\usepackage{upquote}}{}
% use microtype if available
\IfFileExists{microtype.sty}{%
\usepackage{microtype}
\UseMicrotypeSet[protrusion]{basicmath} % disable protrusion for tt fonts
}{}
\usepackage[margin=1in]{geometry}
\usepackage{hyperref}
\hypersetup{unicode=true,
            pdfborder={0 0 0},
            breaklinks=true}
\urlstyle{same}  % don't use monospace font for urls
\usepackage{graphicx,grffile}
\makeatletter
\def\maxwidth{\ifdim\Gin@nat@width>\linewidth\linewidth\else\Gin@nat@width\fi}
\def\maxheight{\ifdim\Gin@nat@height>\textheight\textheight\else\Gin@nat@height\fi}
\makeatother
% Scale images if necessary, so that they will not overflow the page
% margins by default, and it is still possible to overwrite the defaults
% using explicit options in \includegraphics[width, height, ...]{}
\setkeys{Gin}{width=\maxwidth,height=\maxheight,keepaspectratio}
\IfFileExists{parskip.sty}{%
\usepackage{parskip}
}{% else
\setlength{\parindent}{0pt}
\setlength{\parskip}{6pt plus 2pt minus 1pt}
}
\setlength{\emergencystretch}{3em}  % prevent overfull lines
\providecommand{\tightlist}{%
  \setlength{\itemsep}{0pt}\setlength{\parskip}{0pt}}
\setcounter{secnumdepth}{0}
% Redefines (sub)paragraphs to behave more like sections
\ifx\paragraph\undefined\else
\let\oldparagraph\paragraph
\renewcommand{\paragraph}[1]{\oldparagraph{#1}\mbox{}}
\fi
\ifx\subparagraph\undefined\else
\let\oldsubparagraph\subparagraph
\renewcommand{\subparagraph}[1]{\oldsubparagraph{#1}\mbox{}}
\fi

%%% Use protect on footnotes to avoid problems with footnotes in titles
\let\rmarkdownfootnote\footnote%
\def\footnote{\protect\rmarkdownfootnote}

%%% Change title format to be more compact
\usepackage{titling}

% Create subtitle command for use in maketitle
\newcommand{\subtitle}[1]{
  \posttitle{
    \begin{center}\large#1\end{center}
    }
}

\setlength{\droptitle}{-2em}

  \title{}
    \pretitle{\vspace{\droptitle}}
  \posttitle{}
    \author{}
    \preauthor{}\postauthor{}
    \date{}
    \predate{}\postdate{}
  

\begin{document}

The discussion about ethical principles can be found today in many
socially relevant areas. Research on digital technologies such as
autonomous driving, which is already possible and may soon be reality on
our streets, must deal with such principles. In Germany, politicians
adopted the first set of ethical standards for autonomous driving system
manufacturing {[}@threesixty{]}. But the question if and how ``moral
programming'' {[}@birnbacher2016: 9.{]} is possible will be difficult to
solve and the political discussion will remain. Political framing is a
phenomenon which is used more and more by politicians today to win
people's votes and to convince people for such decisions as mentioned in
the example of the autonomous driving. Framing is the setting of frames
of interpretation over language and can work in very different ways.
Frames are not manipulative per se, but a natural part of all language
and one can't resist frames neuronally as they are always activated when
we hear or read a word.

In this paper, the ethical decision-making is examined through a survey
of action decisions with three different treatment groups regarding the
known trolley dilemma and footbridge dilemma. The original dilemma was
developed by {[}@foot1978: 2{]} and consists of a mental experiment in
which the driver a runaway tram is heading towards a group of five
people and will kill these people, unless the tram is diverted to a
second track where only one person is standing. The driver has the
choice of doing nothing, in which case, five people will perish, or
divert the tram, ensuring the death of the person on the other track.
This thought scenario is traditionally employed to evaluate
deontological and consequentialist reasoning mechanisms (cf.~Kahane
2015: 557).

In this paper we will also consider the gender differences in ethical
decision-making, as studies involving trolley dilemmas have found that
men are more inclined to make utilitarian decisions than women.
{[}@zamzow2009variations: 371{]} Research of the past two decades has
focussed on these kind of gender differences with regard to moral
decisions. However, literature reviewing the interplay between gender
and individual ethical positions -- and how it influences moral choices
-- has been ``surprisingly scarce'' {[}@donoho2012gender: 57{]}. This
paper will therefore contribute to existing literature by considering
the role of ethical positions when explaining gender related patterns in
moral decision making.

In the following section, the theoretical background of ethical decision
making, ethical positions and the role of gender is examined. In the
methodological section the online survey experiment is explained and the
operationalization of the variables used in this analysis is discussed.
Afterwards, problems regarding the randomization into three treatment
groups are analysed and descriptive statistics are reported. What
follows is the empirical analysis, for which 12 different models were
estimated. In the last section, the results of the analysis are
discussed in regard to their broader implications for theory and some
concluding remarks on future research are made.


\end{document}
